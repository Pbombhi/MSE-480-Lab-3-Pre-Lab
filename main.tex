%----------------------------------------------------------------------------------------
%   PACKAGES AND DOCUMENT CONFIGURATIONS
%----------------------------------------------------------------------------------------

\documentclass[12pt]{article}
\usepackage[english]{babel}
\usepackage[utf8]{inputenc}
\usepackage{float}

\usepackage{graphicx,epstopdf}   %for embedding images and for conveting eps to pdf
\usepackage{subfig}              %for sub images
\usepackage[margin=1in,includefoot]{geometry}	%changes the margins of report to 1 in
\usepackage{lscape}

\usepackage{rotating}   %used for rotating figures
\usepackage[automake,nonumberlist]{glossaries}    %To use glossary
\usepackage{varwidth}
\usepackage{multicol}   %for making columns
\usepackage{amsmath}
\usepackage{appendix}
\usepackage{pdfpages}
\usepackage{empheq}

\usepackage[framed, numbered]{matlab-prettifier}    %to import matlab code

%----------------------------------------------------------------------------------------
%   Acronym/Glossary
%----------------------------------------------------------------------------------------
\makeglossaries
\loadglsentries{glossary}

%----------------------------------------------------------------------------------------
%   Document Start
%----------------------------------------------------------------------------------------

\begin{document}

\begin{titlepage}

\newcommand{\HRule}{\rule{\linewidth}{0.5mm}} % Defines a new command for the horizontal lines, change thickness here

\center % Center everything on the page
 
%----------------------------------------------------------------------------------------
%   HEADING SECTIONS
%----------------------------------------------------------------------------------------

\textsc{\LARGE MSE 480 Manufacturing Systems}\\[1.5cm] % Org Name
\textsc{\Large }\\[0.5cm] % course name
\textsc{\large }\\[0.5cm] % course title

%----------------------------------------------------------------------------------------
%   TITLE SECTION
%----------------------------------------------------------------------------------------

\HRule \\[0.4cm]
{ \huge \bfseries Lab 3: Robotics 2 - Prelab}\\[0.4cm] % Title of report
\HRule \\[1.5cm]
 
%----------------------------------------------------------------------------------------
%   AUTHOR SECTION
%----------------------------------------------------------------------------------------

\begin{minipage}{0.4\textwidth}
    \begin{flushleft} \large
        \emph{Authors:}\\
        Parshant \textsc{Bombhi}\\
        Klark \textsc{Li}
    \end{flushleft}
\end{minipage}
\hfill
\begin{minipage}{0.4\textwidth}
    \begin{flushright} \large
        \emph{Student ID:} \\
        301255126\\
        301276715
    \end{flushright}
\end{minipage}
\vspace{10mm}
%----------------------------------------------------------------------------------------
%   DATE SECTION
%----------------------------------------------------------------------------------------

{\large \today}\\[2cm] % Date, change the \today to a set date if you want to be precise

%----------------------------------------------------------------------------------------
%   LOGO SECTION
%----------------------------------------------------------------------------------------

\includegraphics[scale=2.0]{MSE-Logo.jpg}\\[1cm] %logo
%----------------------------------------------------------------------------------------

\vfill % Fill the rest of the page with whitespace

\end{titlepage}

%----------------------------------------------------------------------------------------
%   Table of Contents/Table of Figures
%----------------------------------------------------------------------------------------
\pagenumbering{roman} %sets numbering of page to roman
\tableofcontents	%makes table of contents
\addcontentsline{toc}{section}{\numberline{}Table Of Contents}	%adds TOC to TOC

\listoffigures
\addcontentsline{toc}{section}{\numberline{}List of Figures}	%adds list of figures to table of contents

 \listoftables
 \addcontentsline{toc}{section}{\numberline{}List of Tables}

\lstlistoflistings
\addcontentsline{toc}{section}{\numberline{}Listings}

% \printglossary
% \addcontentsline{toc}{section}{\numberline{}Glossary}	%adds glossary to table of contents
\pagebreak
%----------------------------------------------------------------------------------------
%   Main Body
%----------------------------------------------------------------------------------------
\setcounter{page}{1}	%resets the page numbering
\pagenumbering{arabic}	%sets numbering of page to arabic
\setlength{\parskip}{1em}

\section{Introduction}
In this prelab we were to find a inverse kinematics solution to a AL5D robot. We used the coordinate frames assigned in figure \ref{fig:frames} and used the position of the end effector to calculate the angles of the joints.

\begin{figure}[H]
    \centering
    \includegraphics[width=0.8\textwidth]{frames.png}
    \caption{Schematic Diagram of AL5D with Coordinate Frames Assigned}
    \label{fig:frames}
\end{figure}
\noindent
\textbf{Case 1:}\\
\[
T_{0t} =
\begin{bmatrix}
         0 & 0 & 1 & 28.58\\
         0 & -1 & 0 & 0\\
         1 & 0 & 0 & 20.8\\
         0 & 0 & 0 & 1
\end{bmatrix}
\]
\textbf{Case 2:}\\
\[
T_{0t} =
\begin{bmatrix}
         1 & 0 & 0 & 15\\
         0 & -1 & 0 & 5\\
         0 & 0 & -1 & 10\\
         0 & 0 & 0 & 1
\end{bmatrix}
\]
\textbf{Case 3:}\\
\[
T_{0t} =
\begin{bmatrix}
         1 & 0 & 0 & 15\\
         0 & -1 & 0 & 5\\
         0 & 0 & -1 & 1\\
         0 & 0 & 0 & 1
\end{bmatrix}
\]
\textbf{Case 4:}\\
\[
T_{0t} =
\begin{bmatrix}
         1 & 0 & 0 & 15\\
         0 & -1 & 0 & -5\\
         0 & 0 & -1 & 5\\
         0 & 0 & 0 & 1
\end{bmatrix}
\]
The lengths of the arms were given as:
\begin{align*}
    d_1 &= 6.193 cm\\
    L_2 &= 14.605 cm\\
    L_3 &= 18.733 cm\\
    d_5 &= 9.843 cm\\
\end{align*}
\pagebreak
\section{Inverse Kinematics}
We can sub-divide the homogeneous transformation matrix, $T_{0t}$, into components of the orientation and position matrix.
\[
T_{0t} =
\begin{bmatrix}
         R_0^n & P_0^n\\
         0 & 1 \\
\end{bmatrix}
 = 
\begin{bmatrix}
         n & s & a & P_0^n\\
         0 & 0 & 0 & 1\\
\end{bmatrix}
\]
Using the equations provided in class for the spherical wrist we can find $\theta_1$.
\begin{align}
    d_c &= P_0^n - d_5*a\nonumber\\
    \begin{bmatrix}
         d_{cx}\\
         d_{cy}\\
         d_{cz}
\end{bmatrix}
&= 
    \begin{bmatrix}
         P_{x}-d_5*a_1\\
         P_{y}-d_5*a_2\\
         P_{z}-d_5*a_3\nonumber
\end{bmatrix}
\end{align}
We can then use the parameters found to calculate $\theta_1$
\begin{align}
    \theta_1 = tan^{-1}\bigg(\frac{d_{cy}}{d_{cx}}\bigg)
\end{align}
Next we can find $\theta_3$
\begin{align}
    sin(\theta_3) &= \frac{(d_{cz}-d_1)^2+d_{cx}^2+d_{cy}^2-L_2^2-L_3^2}{2L_2L_3}\nonumber\\
    cos(\theta_3) &= \sqrt{(1-(sin(\theta_3))}\nonumber\\
    \theta_3 &= tan^{-1}\bigg(\frac{sin(\theta_3)}{cos(\theta_3)}\bigg)
\end{align}
$\theta_2$ can be found by first finding intermediate matrices A and B:
\begin{align}
    A&=\begin{bmatrix}
             L_2+(L_3*sin(\theta_3)) & L_3*cos(\theta_3)\\
             L_3*cos(\theta_3) & -L_2-(L_3*sin(\theta_3))
    \end{bmatrix}\nonumber\\
    B&=\begin{bmatrix}
             d_{cz}-d_1 & \sqrt(d_{cx}^2+d_{cy}^2)
    \end{bmatrix}\nonumber
\end{align}
Then by dividing the two matrices we get the cosine and sine of the angle. This in turn can be used to find $\theta_2$
\begin{align}
    \begin{bmatrix}
             cos(\theta_2)\\
             sin(\theta_2)
    \end{bmatrix}
    = \frac{A}{B}\nonumber\\
    \theta_2=tan^{-1}\bigg(\frac{sin(\theta_2)}{cos(\theta_2)}\bigg)
\end{align}
Finally we can find $\theta_4$ and $\theta_5$ by:
\begin{align*}
    A_{01} &= 
    \begin{bmatrix}
          cos(\theta_1) & 0 & sin(\theta_1)\\
          sin(\theta_1) & 0 & -cos(\theta_1)\\
          0 & 1 & 0
    \end{bmatrix}\\
    A_{12} &=
    \begin{bmatrix}
             -sin(\theta_2) & -cos(\theta_2) & 0\\
             cos(\theta_2) & -sin(\theta_2) & 0 \\
              0 & 0 & 1
    \end{bmatrix}\\
    A_{23} &=
    \begin{bmatrix}
             sin(\theta_3) & cos(\theta_3) & 0\\
             -cos(\theta_3) & sin(\theta_3) & 0 \\
              0 & 0 & 1
    \end{bmatrix}\\
    R_{03}&=A_{01}A_{12}A_{23}\\
    R_{3t}&=R_{03}*R_{0t}\\
\end{align*}
\begin{align}
    cos(\theta_4) = R_{3t}(1,3)\nonumber\\
    sin(\theta_4) = R_{3t}(2,3)\nonumber\\
    cos(\theta_5) = R_{3t}(3,2)\nonumber\\
    sin(\theta_5) = R_{3t}(3,1)\nonumber\\
    \theta_4 = tan^{-1}\bigg(\frac{sin(\theta_4)}{cos(\theta_4)}\bigg)\\
    \theta_5 = tan^{-1}\bigg(\frac{sin(\theta_5)}{cos(\theta_5)}\bigg)
\end{align}
\pagebreak
\section{Results}
Table \ref{tab:results} lists the joint angles calculated by the methods shows above. The Matlab code for the calculations is shown in the appendix.
% Table generated by Excel2LaTeX from sheet 'Sheet1'
\begin{table}[htbp]
  \centering
  \caption{Joint Angles}
    \begin{tabular}{|p{3.5em}|rrrrr|}
    \hline
    \multicolumn{1}{|r|}{} & \multicolumn{1}{l}{$\theta_1$} & \multicolumn{1}{l}{$\theta_2$} & \multicolumn{1}{l}{$\theta_3$} & \multicolumn{1}{l}{$\theta_4$} & \multicolumn{1}{l|}{$\theta_5$} \\
    \hline
    Case 1 & 0     & -0.0157 & 0.0218 & -0.0061 & 0 \\
    Case 2 & 18.4349 & 11.493 & -13.518 & -87.974 & 18.4349 \\
    Case 3 & 18.4349 & 0.2261 & -32.327 & -57.899 & 18.4349 \\
    Case 4 & -18.435 & 7.8519 & -25.946 & -71.906 & -18.435 \\
    \hline
    \end{tabular}%
  \label{tab:results}%
\end{table}%

\section{References}
[1] MSE 480 Robotics Lab 2.pdf , Dr. Amr Marzouk, Spring 2019

\pagebreak

\appendix
\section{Matlab Code}
\lstinputlisting[style=Matlab-editor, basicstyle=\mlttfamily\scriptsize, caption={Matlab Code for Joint Angles}]{MSE480_Robotics2_Prelab.m}
\end{document}
              
            